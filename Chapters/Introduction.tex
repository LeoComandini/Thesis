\chapter{Introduction}
\label{chpr:intro}
Associate dates with events is the essential medium to write history. 
In the distant past it was appanage of a restricted elite of powerful individuals, with the evolution and progress of society, the amount of people able to write their own history has increased.
This lead to the emergence of different versions: not a crucial concern because, as for several human constructions, history can be a distortion of reality.
Converge to a single version is a tough matter, yet it is grounded into the ability of everyone to write their own history.
Precluding some individuals from the ability to state and record their viewpoint does compromise the natural interaction between history and reality.
In this work we focus on the primary obstacle: the ability of recording arbitrary events.

Using proper semantics, an event is mapped into data, which are then embedded in an item suitable for storing and sharing, finally on the resulting object a date is attached. 
Each step can be performed in a variety of manners and has its own optimal choices to fulfil the given requirements. 
In this work we assume the meaning of the data is given and we refer to the procedure of binding a date to data as timestamping. 
We focus on weaknesses and strengths of every choice, starting from physical to digital timestamping, with particular emphasis on the trust issue.
If a third party is placing the date on the item containing the data, it may behave maliciously, e.g. compromising the data or setting a wrong date. This issue can be addressed using distributed consensus technologies, like Bitcoin, allowing anyone to write his own version of the events: oppressed people have the possibility to record what they witness even in hostile environments, reducing powers and responsibilities of central authorities.

How to make this approach viable and accessible on large scale presents tough technical steps, resulting in best practices \cite{OTSWeb} that have been used to define an open protocol standard. This protocol has emerged as the first (along with very few others) non-financial blockchain-related working application \cite{ESMAresponse}.

Our contribution starts with a deep investigation of an improvement proposal \cite{PoePR, PoeIs} to the standard that allows the inclusion of a timestamp inside a regular bitcoin transaction without increasing its size and hence its cost. 
We aim to provide guidelines to properly understand what is behind this technique and the implications it carries.
Finally we present a practical implementation of this new feature we have developed as integration inside a popular bitcoin open source wallet: users can create timestamps within transactions with no additional charge. 

\section{Structure}
In this work we aim to describe exhaustively foundations, benefits and issues of arising from the new proposed technique. 
It requires to traverse different subjects, mainly cryptography, computer science and distributed systems.
In this section we outline the path we are going to undertake.

In Chapter \ref{chpr:timestamping} we define what a timestamp is and we exploit the essential characteristics of its components: operations and attestations.

In Chapter \ref{chpr:trustless}, after a brief introduction on Bitcoin, we show how it can be used to achieve digital timestamping without relying on trusted third parties.

In Chapter \ref{chpr:state-of-art}, we show the state of the art of trustless digital timestamping, with the open source project OpenTimestamps. 
We provide a description of the standard it defines and the solution to address scalability issues.

In Chapter \ref{chpr:ec-commitments} we plunge into the core of the work, analysing the technique of elliptic curve commitment, with main focus on timestamping applications, particularly \textit{sign-to-contract}.

In Chapter \ref{chpr:s2c} we highlight the practical implications of \textit{sign-to-contract}, both benefits and issues. Finally we show a plugin for a popular open source wallet that implements the technique described.

To conclude, in Chapter \ref{chpr:conclusions}, we summarize what has been discussed and draw attention to which future works can start from the point reached.
